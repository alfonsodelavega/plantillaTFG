\documentclass{article}
\usepackage[utf8]{inputenc}
%\usepackage[spanish,es-tabla]{babel}
\usepackage[spanish,es-tabla]{babel}
\usepackage{graphicx, wrapfig}
\usepackage{anyfontsize}
\usepackage[spanish]{babel}
\usepackage{geometry}
\usepackage{fancyhdr}
\usepackage{setspace}
\usepackage{hyperref}
\usepackage{array}
\usepackage{float}
\usepackage{listings}
\usepackage{xcolor}

\colorlet{punct}{red!60!black}

\colorlet{numb}{magenta!60!black}

\usepackage{xcolor}
\usepackage{vmargin}
\usepackage{pdfpages}
\usepackage{svg}
\usepackage{booktabs}
\usepackage{amsmath} % you need amsmath as the demo includes a use of \eqref
\usepackage{cite}
\usepackage[spanish]{babel}

\definecolor{azure}{rgb}{0.0, 0.5, 1.0}
\definecolor{pblue}{rgb}{0.13,0.13,1}
\definecolor{pgreen}{rgb}{0,0.5,0}
\definecolor{pred}{rgb}{0.9,0,0}
\definecolor{pgrey}{rgb}{0.46,0.45,0.48}
\definecolor{background}{HTML}{EEEEEE}
\definecolor{delim}{RGB}{20,105,176}
\hypersetup{
    colorlinks=true,
    linkcolor=azure,
    filecolor=magenta,      
    urlcolor=azure,
    citecolor=azure,
    hypertexnames=true,
}

\lstset{language=Java,
  showspaces=false,
  showtabs=false,
  breaklines=true,
  showstringspaces=false,
  breakatwhitespace=true,
  commentstyle=\color{pgreen},
  keywordstyle=\color{pblue},
  stringstyle=\color{pred},
  basicstyle=\ttfamily,
  moredelim=[il][\textcolor{pgrey}]{},
  moredelim=[is][\textcolor{pgrey}]{\%\%}{\%\%}
}
\lstdefinelanguage{json}{
    basicstyle=\normalfont\ttfamily,
    numbers=left,
    numberstyle=\scriptsize,
    stepnumber=1,
    numbersep=8pt,
    showstringspaces=false,
    breaklines=true,
    frame=lines,
    backgroundcolor=\color{background},
    literate=
     *{0}{{{\color{numb}0}}}{1}
      {1}{{{\color{numb}1}}}{1}
      {2}{{{\color{numb}2}}}{1}
      {3}{{{\color{numb}3}}}{1}
      {4}{{{\color{numb}4}}}{1}
      {5}{{{\color{numb}5}}}{1}
      {6}{{{\color{numb}6}}}{1}
      {7}{{{\color{numb}7}}}{1}
      {8}{{{\color{numb}8}}}{1}
      {9}{{{\color{numb}9}}}{1}
      {:}{{{\color{punct}{:}}}}{1}
      {,}{{{\color{punct}{,}}}}{1}
      {\{}{{{\color{delim}{\{}}}}{1}
      {\}}{{{\color{delim}{\}}}}}{1}
      {[}{{{\color{delim}{[}}}}{1}
      {]}{{{\color{delim}{]}}}}{1},
}


\setpapersize{A4}
\setmargins{3cm}        % margen izquierdo
{1.5cm}                   % margen superior
{15cm}                  % anchura del texto
{22.75cm}               % altura del texto
{1cm}                   % altura de los encabezados
{1cm}                 % espacio entre el texto y los encabezados
{0pt}                   % altura del pie de página
{1cm}                   % espacio entre el texto y el pie de página


\begin{document}
	\begin{titlepage}
		{\fontfamily{phv}\selectfont
			\begin{center}
				\vspace*{-2.5cm}
				
				\includegraphics[width=0.25\textwidth]{images/logoUC}
				
				\vspace{0.2cm}
				\textbf{\Huge\emph{Facultad \\
						de\\
						Ciencias}}
				
				\vspace{1cm}
				\textbf{\fontsize{20}{24}\selectfont Título en español}\\
				\fontsize{19}{24}\selectfont {Título en inglés}
				
				\vspace{1.5cm}
				\fontsize{14}{17}\selectfont Trabajo de Fin de Grado\\
				para acceder al\\
                \vspace{0.5cm}
				\textbf{\fontsize{17}{20}\selectfont GRADO EN INGENIERÍA INFORMÁTICA}
				
				\vfill
				
				
				\vspace{0.8cm}
				
				
				\begin{flushright}
					\fontsize{14}{17}
					Autor: Nombre autor\\
					Director: Nombre director\\
					Codirector: Nombre codirector (opcional)\\
					Convocatoria: fecha (e.g. Julio - 2022)
				\end{flushright}
				
				
		\end{center}}
	\end{titlepage}


% Acknowledgements Page (Optional)                              
\newpage
\begin{center}
{\bf \Huge Agradecimientos}
\end{center}
\vspace{1cm}
\setlength{\baselineskip}{0.8cm}
\onehalfspacing
\begin{flushleft}

\input{agradecimientos.tex}

\end{flushleft}

\newpage

\thispagestyle{empty}

\begin{center}
{\bf \Huge Resumen}
\end{center}
\onehalfspacing

\input{resumen.tex}

\begin{center}
{\bf \Large Palabras clave}
\end{center}
% Insertar pablabras clave  
Palabra clave 1, palabra clave 2, ...

\newpage

\thispagestyle{empty}

\begin{center}
{\bf \Huge Abstract}
\end{center}

\input{abstract.tex}

\begin{center}
{\bf \Large Keywords}
\end{center}
Keyword 1, Keyword 2, ...

\newpage


%\setcounter{page}{1}

	

\tableofcontents
\newpage
\listoffigures
\newpage
\listoftables
\newpage
\newpage

\section{Introducción}

Esta plantilla se ha desarrollado con el fin de facilitar la realización de la memoria del Trabajo Fin de Grado a los futuros egresados de la Facultad de Ciencias.

La siguiente sección muestra las funcionalidades básicas de latex que se aconsejan utilizar durante la escritura del manuscrito.

Si bien se puede modificar el estilo de esta plantilla como se desee, el objetivo de la misma es justo el contrario: que los estudiantes se centren en depurar el contenido de sus memorias, y no pierdan el tiempo con aspectos de presentación de las mismas si así lo desean.
\newpage

\input{latex.tex}
\newpage


\newpage
\bibliographystyle{IEEEtran}
\bibliography{bibliografia.bib}


\end{document}
